\chapter{Introducción}

Este documento contiene la memoria de la práctica obligatoria de laboratorio correspondiente a la asignatura de \emph{Sistemas Distribuidos} del Grado de Ingeniería Informática. El propósito de la practica es el desarrollo de un software que implemente un sistema de almacenamiento de ficheros en la nube usando Java RMI.

El sistema esta formado por tres elementos principales:

\begin{itemize}
\item El servidor, levanta tres servicios, el servicio de datos para uso interno del servidor, el servicio de autenticación mediante el cual podremos registrar nuevos usuarios o comprobar si los usuarios del sistema están correctamente autenticados y finalmente el servicio gestor que se utilizará para gestionar las operaciones del sistema.

\item El repositorio publica dos objetos distribuidos para uso de los clientes o del servidor, estos objetos ofrecen los métodos necesarios para almacenar los ficheros en su sistema local de ficheros, enviarlos al sistema local de ficheros de los clientes, borrar los ficheros del repositorio o crear las carpetas necesarias para almacenar los ficheros de los clientes.

\item El cliente sera la parte utilizada por el usuario para gestionar sus ficheros, publicara a su vez un objeto distribuido que se utilizara para almacenar ficheros en el disco local del cliente.
\end{itemize}

La interfaz de la aplicación se implementa mediante un menú en modo texto, los resultados de las operaciones se volcaran sobre la salida estándar. De cara a una depuración más eficiente del sistema se generará un fichero de log llamado ``log.txt"\ en la carpeta donde se ejecuten los procesos donde se podrá encontrar información adicional para depurar la aplicación.