%% Los cap'itulos inician con \chapter{T'itulo}, estos aparecen numerados y
%% se incluyen en el 'indice general.
%%
%% Recuerda que aqu'i ya puedes escribir acentos como: 'a, 'e, 'i, etc.
%% La letra n con tilde es: 'n.

\chapter{Introducción}

Este documento contiene la memoria de la práctica obligatoria de laboratorio correspondiente a la asignatura de \emph{Sistemas Distribuidos} del Grado de Ingeniería Informática. El propósito de la practica es el desarrollo de un software que implemente un sistema de almacenamiento de ficheros en la nube usando Java RMI.

El aplicativo consiste en tres ejecutables diferentes, el servidor levanta tres servicios, dos de ellos seran usados por los distintos usuarios del servicio distribuido mientras que el tercer servicio sera para uso interno de los servicios propios del servidor.

El repositorio publica dos objetos distribuidos para uso de los clientes o del servidor, estos objetos ofrecen los metodos necesarios para almacenar los ficheros en su sistema local de ficheros o enviarlos al sistema local de ficheros de los clientes.

El cliente sera la parte utilizada por el usuario para gestionar sus ficheros, publicara a su vez un objeto distribuido que se utilizara para almacenar ficheros en el disco local del cliente.
