
\chapter{Introducción}

Este documento contiene la memoria de la práctica obligatoria de laboratorio correspondiente a la asignatura de \emph{Sistemas Distribuidos} del Grado de Ingeniería Informática. El propósito de la practica es el desarrollo de un software que implemente un sistema de almacenamiento de ficheros en la nube usando Java RMI.

El aplicativo consiste en tres ejecutables diferentes, el servidor levanta tres servicios, dos de ellos serán usados por los distintos usuarios del servicio distribuido mientras que el tercer servicio sera para uso interno de los servicios propios del servidor.

El repositorio publica dos objetos distribuidos para uso de los clientes o del servidor, estos objetos ofrecen los métodos necesarios para almacenar los ficheros en su sistema local de ficheros o enviarlos al sistema local de ficheros de los clientes.

El cliente sera la parte utilizada por el usuario para gestionar sus ficheros, publicara a su vez un objeto distribuido que se utilizara para almacenar ficheros en el disco local del cliente.

La interfaz de la aplicación se implementa mediante un menú en modo texto, los resultados de las operaciones se volcaran sobre la salida estándar. De cara a una depuración más eficiente del sistema se generará un fichero de log llamado ``log.txt"\ en la carpeta donde se ejecuten los procesos.

