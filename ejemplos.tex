\chapter{Ejemplos de uso}

En este capitulo mostraremos ejemplos de uso de la aplicación, para los ejemplos he registrado en el servidor cuatro repositorios y siete clientes, cada cliente tendra asociados varios ficheros.
Este sera el diagrama de los elementos en las pruebas.
\includegraphics[width=0.8\linewidth]{escenario_pruebas.png}


\section{Operaciones del servidor}
\begin{lstlisting}[language=bash,frame=single,texcl=true,basicstyle=\small]
============Menu Servidor============
1-Listar Clientes
2-Listar Repositorios
3-Listar Parejas Repositorio-Cliente
4-Salir
============Menu Servidor============
Seleccione una opcion: 1
Cliente: 0-cliente1
Cliente: 1-cliente2
Cliente: 2-cliente3
Cliente: 3-cliente4
Cliente: 4-cliente5
Cliente: 5-cliente6
Cliente: 6-cliente7
============Menu Servidor============
1-Listar Clientes
2-Listar Repositorios
3-Listar Parejas Repositorio-Cliente
4-Salir
============Menu Servidor============
Seleccione una opcion: 2
-- Repositorios registrados en el sistema: 
Repositorio: 0-repositorio1
Repositorio: 1-repositorio2
Repositorio: 2-repositorio3
Repositorio: 3-repositorio4
============Menu Servidor============
1-Listar Clientes
2-Listar Repositorios
3-Listar Parejas Repositorio-Cliente
4-Salir
============Menu Servidor============
Seleccione una opcion: 3
Cliente: 0-cliente1<->0-repositorio1
Cliente: 1-cliente2<->1-repositorio2
Cliente: 2-cliente3<->2-repositorio3
Cliente: 3-cliente4<->3-repositorio4
Cliente: 4-cliente5<->0-repositorio1
Cliente: 5-cliente6<->1-repositorio2
Cliente: 6-cliente7<->2-repositorio3
============Menu Servidor============
1-Listar Clientes
2-Listar Repositorios
3-Listar Parejas Repositorio-Cliente
4-Salir
============Menu Servidor============
Seleccione una opcion: 
\end{lstlisting}


\section{Operaciones del repositorio}
Este es un ejemplo de las operaciones disponibles en los repositorios.

\section{Operación compartir fichero}
Ejemplo de la compartición de un fichero de usuario.

\section{Operación subir fichero}
Un usuario podra subir un fichero siguiendo estos pasos.

\section{Operación descargar fichero}
Un usuario podra descargar un fichero de la nube siguiendo estos pasos.
