\chapter{Ejemplos de uso}

En este capitulo mostraremos ejemplos de uso de la aplicación, para los ejemplos he registrado en el servidor cuatro repositorios y siete clientes, cada cliente tendra asociados varios ficheros.
Este sera el diagrama de los elementos en las pruebas.
\includegraphics[width=0.8\linewidth]{escenario_pruebas.png}

\section{Operaciones del servidor}
\begin{lstlisting}[language=bash,frame=single,texcl=true,basicstyle=\small]
============Menu Servidor============
1-Listar Clientes
2-Listar Repositorios
3-Listar Parejas Repositorio-Cliente
4-Salir
============Menu Servidor============
Seleccione una opcion: 1   
-- Clientes registrados en el sistema --
Cliente: 0-cliente1
Cliente: 1-cliente2
Cliente: 2-cliente3
Cliente: 3-cliente4
Cliente: 4-cliente5
Cliente: 5-cliente6
Cliente: 6-cliente7
============Menu Servidor============
1-Listar Clientes
2-Listar Repositorios
3-Listar Parejas Repositorio-Cliente
4-Salir
============Menu Servidor============
Seleccione una opcion: 2
-- Repositorios registrados en el sistema --
Repositorio: 0-repositorio1
Repositorio: 1-repositorio2
Repositorio: 2-repositorio3
Repositorio: 3-repositorio4
============Menu Servidor============
1-Listar Clientes
2-Listar Repositorios
3-Listar Parejas Repositorio-Cliente
4-Salir
============Menu Servidor============
Seleccione una opcion: 3
-- Clientes - repositorio registrados en el sistema --
Cliente: 0-cliente1 -> 0-repositorio1
Cliente: 1-cliente2 -> 1-repositorio2
Cliente: 2-cliente3 -> 2-repositorio3
Cliente: 3-cliente4 -> 3-repositorio4
Cliente: 4-cliente5 -> 0-repositorio1
Cliente: 5-cliente6 -> 1-repositorio2
Cliente: 6-cliente7 -> 2-repositorio3
============Menu Servidor============
1-Listar Clientes
2-Listar Repositorios
3-Listar Parejas Repositorio-Cliente
4-Salir
============Menu Servidor============
Seleccione una opcion: 
\end{lstlisting}


\section{Operaciones del repositorio}
Este es un ejemplo de las operaciones disponibles en los repositorios.
\subsection{Registro nuevo repositorio}
\begin{lstlisting}[language=bash,frame=single,texcl=true,basicstyle=\small]
============Menu Repositorio============
1-Registrar nuevo repositorio
2-Autenticarse en el sistema
3-Salir
============Menu Repositorio============
Seleccione una opcion: 1
Introduzca el usuario: repositorio1
Introduzca la contraseña: rep1
Introduzca la contraseña de nuevo: rep1
============ repositorio0 ============
1-Listar Clientes
2-Listar ficheros del Cliente
3-Salir
============ repositorio0 ============
Seleccione una opcion: 
\end{lstlisting}
\subsection{Operaciones repositorio}
\begin{lstlisting}[language=bash,frame=single,texcl=true,basicstyle=\small]
============ repositorio0 ============
1-Listar Clientes
2-Listar ficheros del Cliente
3-Salir
============ repositorio0 ============
Seleccione una opcion: 1   
Listado de clientes en el repositorio
Cliente: 0-cliente1
Cliente: 4-cliente5
============ repositorio0 ============
1-Listar Clientes
2-Listar ficheros del Cliente
3-Salir
============ repositorio0 ============
Seleccione una opcion: 2
Clientes validos: 
0 4 
Seleccione un cliente: 0
Fichero: test1.txt
Fichero: test2.txt
============ repositorio0 ============
1-Listar Clientes
2-Listar ficheros del Cliente
3-Salir
============ repositorio0 ============
Seleccione una opcion: 
\end{lstlisting}

\section{Cliente}
Ejemplo de uso del cliente
\subsection{Registro nuevo cliente}
\begin{lstlisting}[language=bash,frame=single,texcl=true,basicstyle=\small]
============Menu Cliente============
1-Registrar nuevo cliente
2-Autenticarse en el sistema
3-Salir
============Menu Cliente============
Seleccione una opcion: 1
Introduzca usuario: cliente3
Introduzca contraseña: cli3
Introduzca la contraseña de nuevo: cli3
============-cliente3-============
1-Subir fichero
2-Bajar fichero
3-Borrar fichero
4-Compartir ficheros
5-Listar ficheros
6-Listar clientes del sistema
7-Salir
============-cliente3-============
Seleccione una opcion: 
\end{lstlisting}
\subsection{Operación subir fichero}
\begin{lstlisting}[language=bash,frame=single,texcl=true,basicstyle=\small]
============-cliente3-============
1-Subir fichero
2-Bajar fichero
3-Borrar fichero
4-Compartir ficheros
5-Listar ficheros
6-Listar clientes del sistema
7-Salir
============-cliente3-============
Seleccione una opcion: 1
Introduzca fichero a enviar: test1.txt
Fichero enviado correctamente a la nube
============-cliente3-============
1-Subir fichero
2-Bajar fichero
3-Borrar fichero
4-Compartir ficheros
5-Listar ficheros
6-Listar clientes del sistema
7-Salir
============-cliente3-============
Seleccione una opcion:
\end{lstlisting}
\subsection{Operación descargar fichero}
\begin{lstlisting}[language=bash,frame=single,texcl=true,basicstyle=\small]
============-cliente3-============
1-Subir fichero
2-Bajar fichero
3-Borrar fichero
4-Compartir ficheros
5-Listar ficheros
6-Listar clientes del sistema
7-Salir
============-cliente3-============
Seleccione una opcion: 2
test1.txt 
Seleccione el fichero que quiere recibir: test1.txt
Escribiendo fichero en test1.txt
============-cliente3-============
1-Subir fichero
2-Bajar fichero
3-Borrar fichero
4-Compartir ficheros
5-Listar ficheros
6-Listar clientes del sistema
7-Salir
============-cliente3-============
Seleccione una opcion: 
\end{lstlisting}
\subsection{Operación borrar un fichero}
\begin{lstlisting}[language=bash,frame=single,texcl=true,basicstyle=\small]
============-cliente3-============
1-Subir fichero
2-Bajar fichero
3-Borrar fichero
4-Compartir ficheros
5-Listar ficheros
6-Listar clientes del sistema
7-Salir
============-cliente3-============
Seleccione una opcion: 3
test1.txt 
Seleccione el fichero que quiere eliminar: test1.txt
============-cliente3-============
1-Subir fichero
2-Bajar fichero
3-Borrar fichero
4-Compartir ficheros
5-Listar ficheros
6-Listar clientes del sistema
7-Salir
============-cliente3-============
Seleccione una opcion:
\end{lstlisting}
\subsection{Otras operaciones}
\begin{lstlisting}[language=bash,frame=single,texcl=true,basicstyle=\small]
============-cliente1-============
1-Subir fichero
2-Bajar fichero
3-Borrar fichero
4-Compartir ficheros
5-Listar ficheros
6-Listar clientes del sistema
7-Salir
============-cliente1-============
Seleccione una opcion: 5   
Fichero: test1.txt
Fichero: test2.txt
============-cliente1-============
1-Subir fichero
2-Bajar fichero
3-Borrar fichero
4-Compartir ficheros
5-Listar ficheros
6-Listar clientes del sistema
7-Salir
============-cliente1-============
Seleccione una opcion: 6
Cliente: 0-cliente1
Cliente: 1-cliente2
Cliente: 2-cliente3
Cliente: 3-cliente4
Cliente: 4-cliente5
Cliente: 5-cliente6
Cliente: 6-cliente7
============-cliente1-============
1-Subir fichero
2-Bajar fichero
3-Borrar fichero
4-Compartir ficheros
5-Listar ficheros
6-Listar clientes del sistema
7-Salir
============-cliente1-============
Seleccione una opcion:
\end{lstlisting}
