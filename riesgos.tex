\chapter{Riesgos, consideraciones y conclusiones}

El proceso para desarrollo de la practica ha sido un proceso incremental por lo que algunas decisiones de diseño pueden no ser todo lo correctas, en el código fuente entregado esta también el histórico y la evolución del código, el histórico se puede consultar usando la herramienta de control de versiones \textbf{git}.

\section{Revisión de decisiones de diseño}

\begin{enumerate}\label{sec:consideraciones_datos}
	\item La clase cliente del paquete común necesita una refactorización, en muchos métodos remotos estamos enviando toda la información del cliente, eso no esta optimizado y en sucesiones iteraciones del desarrollo habría que optimizarlo para enviar solo la información necesaria
	\item El sistema no esta llevando un control estricto de si los elementos están online o no, este control debería llevarse en el servicio de datos con un flag que indique si el cliente o el repositorio esta online, este flag habría que marcarlo cuando se hace el login o se registra el cliente/repositorio y desmarcarlo al salir de la aplicación. También habría que hacer un pooling periódico desde el servidor para detectar que sistemas están activos y cuales no. Otro posible enfoque es que si falla alguna de las operaciones remotas se pueda hacer una llamada al servicio para verificar si el servicio esta activo o no y marcarlo en este momento.
	\item El repositorio crea una carpeta con su id para almacenar los ficheros, el motivo es facilitar las pruebas con varios repositorios lanzados desde la misma carpeta.
	\item Con el desarrollo actual todos los objetos distribuidos tienen que estar en el mismo registro RMI en localhost:8888, he intentado abstraer las operaciones RMI a una clase externa de las implementaciones (clase Utils) de los servicios por si se plantea la posibilidad de modificar el sistema de forma que las url de los objetos distribuidos puedan estar en registros de host diferentes. Creo que todavía queda trabajo pendiente para sacar todos los lookup del código de la implementación pero por motivos de tiempo no ha sido posible desarrollarlo, entraría en la lista de tareas por resolver.
\end{enumerate}
