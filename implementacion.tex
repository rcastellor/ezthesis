\chapter{Detalles de implementación}

\section{Estructura de directorios}

Las Carpeta comun contiene las clases e interfaces comunes al resto del proyecto.

\begin{compactitem}
	\item \textbf{Cliente}, clase con la información de los clientes, el usuario, el id y listas con los ficheros del cliente
	\item \textbf{Fichero}, clase proporcionada por el equipo docente para enviar los ficheros
	\item \textbf{GUI}, clase con métodos genéricos para mostrar por pantalla información y recoger las entradas del usuario
	\item \textbf{Logger}, clase con métodos para volcar a un fichero log de la aplicación y de esta forma no mezclarse con la interfaz del usuario
	\item \textbf{Metadato}, clase con información referente a los ficheros, tendrá el repositorio, el id del cliente propietario del fichero, etc.
	\item \textbf{Repositorio}, clase con datos de los repositorios registrados en el sistema, los id, el usuario y las url de los objetos remotos que publica el repositorio.
	\item \textit{ServicioAutenticacionInterface}, interfaz del servicio de autenticación.
	\item \textit{ServicioDatosInterface}, interfaz del servicio de datos
	\item \textit{ServicioGestorInterface}, interfaz del servicio gestor
	\item \textit{ServicioClOperadorInterface}, interfaz del servicio cliente-operador
	\item \textit{ServicioSrOperadorInterface}, interfaz del servicio servidor-operador
	\item \textit{ServicioDiscoClienteInterface}, interfaz del servicio de disco del cliente
	\item \textbf{Utils}, clase con métodos genéricos para el uso por parte de la aplicación
\end{compactitem}

La carpeta servidor, contiene el código necesario para lanzar el servidor y la implementación del servicio de datos, el servicio de autenticación y el servicio gestor.

\begin{compactitem}
	\item \textbf{Servidor}, clase que lanza el servidor, crea los objetos remotos y los publica en el registro RMI
	\item \textbf{ServicioAutenticacionImpl}, clase que implementa el servicio de autenticación
	\item \textbf{ServicioDatosImpl}, clase que implementa el servicio de datos
	\item \textbf{ServicioGestorImpl}, clase que implementa el servicio gestor
\end{compactitem}

La carpeta repositorio contiene el código necesario para lanzar los repositorios y la implementación de los servicios operador-cliente y operador-servidor.

\begin{compactitem}
	\item \textbf{Repositorio}, clase que lanza el repositorio y genera el menú para operar con el, publica los objetos remotos.
	\item \textbf{ServicioClOperadorImpl}, clase que implementa el servicio cliente-operador
	\item \textbf{ServicioSrOperadorImpl}, clase que implementa el servicio servidor-operador
\end{compactitem}

La carpeta cliente contiene el código necesario para lanzar los clientes y la implementación de la interfaz remota del servicio de disco del cliente

\begin{compactitem}
	\item \textbf{Cliente}, clase que lanza el cliente, genera el menú para poder operar con el y publica los objetos remotos
	\item \textbf{ServicioDiscoClienteImpl}, clase que implementa el servicio de disco del cliente a través del cual se recibirán ficheros de la nube
\end{compactitem}

\section{Modelo de datos}

El modelo de datos esta basado en la api Collections de Java.

Cada elemento del sistema sera identificado por un id numérico que formara parte del objeto como atributo.

En el servicio de datos se almacenara la siguiente información

Lista con todos los clientes registrados en la aplicación.
Relación de nombre de usuario con id de clientes con datos de cliente en una HashMap<String, Integer> de la que se podrá obtener el id de un cliente dado su nombre de usuario.
Relación de id de usuario con el objeto de tipo Cliente que tiene toda la información referente a cada cliente.
Variable sesioncli con el ultimo identificador de cliente asignado, a cada cliente nuevo se le asignara como identificador el valor de esta variable y esta se incrementara en espera del próximo cliente.

La clase Cliente del paquete comun tiene la información propia del cliente y dos mapas indexadas por el nombre del fichero, una con los metadatos de los ficheros cuyo propietario es el cliente y otra con los metadatos de los ficheros que se han compartido con el cliente. Las operaciones con los datos de los ficheros se realizaran mediante llamadas a metodos de esta clase y en ningún momento accediendo directamente a las collection que albergan los datos.
